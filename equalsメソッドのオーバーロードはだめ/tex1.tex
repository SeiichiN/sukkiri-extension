\documentclass[uplatex, dvipdfmx]{jsarticle}

\include{begin}

\section{equals()メソッドのオーバーライド}

\subsection{準備}

以下のような Userクラスを考える。

\begin{lstlisting}[caption=User.java]
public class User {
  public String name;

  public User(String name) {
    this.name = name;
  }
}
\end{lstlisting}

\subsection{equals()メソッドのオーバーロード}

これに、以下のような equals()メソッドをつくる。
これは絶対にやってはいけないと言われているが、
Objectクラスの equals()メソッドをオーバーライドするのではなくて、
オーバーロードしてみる。

\begin{lstlisting}[caption=User.java]
...( つづき )...

  public boolean equals(User u) {
    if (this.name.equals(u.name)) {
      return true;
    }
    return false;
  }
\end{lstlisting}

テストのために、以下のようなコードを試してみると\dots\dots\dots。

\begin{lstlisting}[caption=Main.java]
public class Main {
  public static void main(String[] args) {
    User u1 = new User(``Sasuke'');
    User u2 = new User(``Sasuke'');
    System.out.println(u1.equals(u2));
  }
}
\end{lstlisting}

これは、\fbox{ true } と表示される。
では、これは?

\begin{lstlisting}[caption=Main.java]
public class Main {
  public static void main(String[] args) {
    List<User> userList = new ArrayList<>();
    User u1 = new User(``Sasuke'');
    userList.add(u1);
    u1 = new User(``Sasuke'');
    userList.remove(u1);
    for (User u : userList) {
      System.out.println(u.name);
    }
    System.out.println(userList.size());
  }
}
\end{lstlisting}

これは \fbox{ Sasuke } / \fbox{ 1 } と表示される。
この場合、userList に add() した h1 と、
userList から remove() した h1 を同じものとして見なしていない
ことがわかる。

\begin{tcolorbox}
 equals()メソッドのオーバーロードはバグの元となるので、絶対やらないように
 と言われている。
\end{tcolorbox}

\subsection{equals()メソッドのオーバーライド}

User.java の中の equals()メソッドを以下のように
Objectクラスの equals()メソッドをオーバーライドしたものにすると、
結果はちがってくる。

\begin{lstlisting}[caption=User.java]
...( つづき )...

  @Override
  public boolean equals(Object o) {
    if (this == o) {
      return true;
    }
    if (!(o instanceof User)) {
      return false;
    }
    User u = (User) o;
    if (this.name.equals(u.name)) {
      return true;
    }
    return false;
  }
\end{lstlisting}

このコードに変えてから Main.java を実行すると

\fbox{ 0 } となる。

\section{hashCode()メソッドのオーバーライド}

equals()メソッドをオーバーライドしたら、hashCode()メソッドも
オーバーライドしなければならない。

\subsection{hashCode()をオーバーライドしない場合}

先ほどの equals()メソッドをオーバーライドした Userクラスを
使ってみる。

以下のようなコードを試す。

\begin{lstlisting}[caption=Main.java]
import java.util.HashSet;
import java.util.Set;

public class Main {

  public static void main(String[] args) {
    Set<User> userSet = new HashSet<>();
    User u = new User("Sasuke");
    userSet.add(u);
    showList(userSet);
    u = new User("Sasuke");
    userSet.remove(u);
    showList(userSet);

  }

  public static void showList(Set<User> list) {
    for (User u : list) {
      System.out.println(u.name);
    }
    System.out.println("要素数:" + list.size());
  }
} 
\end{lstlisting}

このコードを実行すると、以下のように出力される。

\begin{lstlisting}[label=console, numbers=none]
Sasuke
要素数:1
Sasuke
要素数:1
\end{lstlisting}
    
途中で remove()されているのにもかかわらず、remove()されていない。

\subsection{hashCode()メソッドのオーバーライド}

そこで、以下のように hashCode()メソッドをオーバーライドする。

\begin{lstlisting}[caption=User.java]
...( つづき )...

  @Override 
  public int hashCode() {
    int result = 37;
    result = result * 31 + name.hashCode();
    return result;
  }
}
\end{lstlisting}

こののち、先ほどの Mainクラスを実行すると、以下のように出力される。

\begin{lstlisting}[label=console, numbers=none]
Sasuke
要素数:1
要素数:0
\end{lstlisting}




\include{end}

%% 修正時刻: Wed 2022/06/08 14:47:060
